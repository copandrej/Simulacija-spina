\documentclass[12pt, a4paper]{article}
\usepackage[utf8]{inputenc}
\title{  Domača naloga\\\large\textbf{Simulacija spina v časovno odvisnem polju}\\
\large Računalniške tehnologije}
\usepackage{amsmath} 
\usepackage{braket}
\usepackage{listings}

\author{Andrej Čop}
\date{Maj 2022}

\begin{document}
\maketitle
\newpage

\tableofcontents
\newpage

\section{Uvod}
V tej domači nalogi raziskujem dinamiko in precesijo spina v časovno spremenljivem polju. Najprej poiščem splošno rešitev za gibanje spina v polju v smeri $x$ in $z$. Nato dinamiko spina v časovno spremenljivem polju, kjer za nek čas $T$ deluje polje v $z$ smeri, nato pa za čas $bT$ deluje polje v smeri osi $x$. Napišem program v katerem simuliram pred tem opisano dinamiko po daljšem času in v odvisnosti od parametra $b$.
%komentar
\section{Izpeljava enačbe}
Po razdelku 3.75 iz učbenika \textit{Kvantne in računalniške tehnologije}, kjer je izpeljan Hamiltonov operator za elektronski spin v polju, lahko poiščemo splošno rešitev za gibanje spina če je magneto polje samo v smeri $x$ ali samo v smeri $z$.\\
Za smer $z$ je v učbeniku izpeljava že prikazana. Podobno naredimo v v smeri $x$:\\
\begin{equation}
H = \frac{h\omega}{2}
\begin{bmatrix}
0 & 1\\
1 & 0
\end{bmatrix}
\end{equation}\\
Lastni stanji sta kar $\ket{+}$ in $\ket{-}$ in časovnin razvoj je:


\begin{equation}
\ket{\psi(t)} = \cos\frac{\phi}{2}e^{-i(\phi+\omega_0 t)/2}\ket{+} + \sin\frac{\theta}{2}e^{i(\theta+\omega_0 t)/2}\ket{+}
\end{equation}\\
Kroženje(precesija) spina po \textit{Blochovi sferi}, imenovana tudi \textit{Larmorjeva precesija}, je tako podobna kot, če je polje v smeri $z$.

\section{Časovno odvisno polje}
Obravnavamo gibanje spina v časovno spremenljivem polju. S pomočjo programa, ki bo simuliral spin v kratkih časovnih intervalih in pri tem za čas $T$ uporabljamo enačbo za polje v smeri $z$ in nato za nek čas $bT$ v smeri $x$. 
\subparagraph{Hipoteza:}
Predvidevam, da za dovolj majhen $T$ in $b$ (tj. T precej manjši od obhodnega časa), bo rezultat simulacije precesija spina v obliki spirale na \textit{Blochovi sferi}

\section{Program za simuliranje}
Za simuliranje uporabljam programski jezik \textit{python} in module \textit{numpy, matplotlib}, \textit{cmath}, ter za simuliranje kvantnih sistemov \textit{qutip}.\\
Najprej nastavimo parametre in bazne vektorje:
\lstset{language=Python} 
\begin{lstlisting}
t = 0.05
T = math.pi
time = theta = phi = 0
frekvenca = 1 

upSpin = basis(2, 0)
downSpin = basis(2, 1)

plus = (upSpin + downSpin)/math.sqrt(2)  # | + >
minus = (upSpin - downSpin)/math.sqrt(2)  # | - >
\end{lstlisting}
Funkcija za računanje vektorja v odvisnosti od časa (za poljubno bazo, kot parameter):
\begin{lstlisting}
def Psi(T, base1, base2, phi, theta):
    scalar = (-1 * (1j) * (phi + frekvenca * T)) / 2
    n = cmath.cos(theta / 2) * cmath.e ** scalar
    base1 = base1 * n
    scalar = ((1j * (phi + frekvenca * T)) / 2)
    n = cmath.sin(theta / 2) * cmath.e ** scalar
    base2 = base2 * n
    vec = base2 + base1
    return vec
\end{lstlisting}
Funkcijo poganjamo na majhnih časovnih intervalih in pri tem posodabljamo \textit{phi} in \textit{theta}: $\phi(t) = \phi + \omega_0t$
\section{Rezultati simulacije}

\section{Zaključek}

\section{Literatura}

\end{document}