\documentclass[12pt, a4paper]{article}
\usepackage[utf8]{inputenc}
\title{  Domača naloga\\\large\textbf{Simulacija spina v časovno odvisnem polju}\\
\large Računalniške tehnologije}
\usepackage{amsmath} 
\usepackage{braket}

\author{Andrej Čop}
\date{Maj 2022}

\begin{document}
\maketitle
\newpage

\tableofcontents
\newpage

\section{Uvod}
V tej domači nalogi raziskujem dinamiko in precesijo spina v časovno spremenljivem polju. Najprej poiščem splošno rešitev za gibanje spina v polju v smeri $x$ in $z$. Nato dinamiko spina v časovno spremenljivem polju, kjer za nek čas $T$ deluje polje v $z$ smeri, nato pa za čas $bT$ deluje polje v smeri osi $x$. Napišem program v katerem simuliram pred tem opisano dinamiko po daljšem času in v odvisnosti od parametra $b$.
%komentar
\section{Izpeljava enačbe}
Po razdelku 3.75 iz učbenika \textit{Kvantne in računalniške tehnologije}, kjer je izpeljan Hamiltonov operator za elektronski spin v polju, lahko poiščemo splošno rešitev za gibanje spina če je magneto polje samo v smeri $x$ ali samo v smeri $z$.\\
Za smer $z$ je v učbeniku izpeljava že prikazana. Podobno naredimo v v smeri $x$:\\
\begin{equation}
H = \frac{h\omega}{2}
\begin{bmatrix}
0 & 1\\
1 & 0
\end{bmatrix}
\end{equation}\\
Lastni stanji sta kar $\ket{+}$ in $\ket{-}$ in časovnin razvoj je:


\begin{equation}
\ket{\psi(t)} = \cos\frac{\phi}{2}e^{-i(\phi+\omega_0 t)/2}\ket{+} + \sin\frac{\theta}{2}e^{i(\theta+\omega_0 t)/2}\ket{+}
\end{equation}\\
Kroženje(precesija) spina po Blochovi sferi, imenovana tudi \textit{Larmorjeva precesija}, je tako podobno kot, če je polje v smeri $z$.

\section{Časovno odvisno polje}
Obravnavamo givanje spina v časovno spremenljivem polju. S pomočjo programa, ki bo simuliral spin v kratkih časovnih intervalih in pri tem za čas $T$ uporabljamo enačbo za polje v smeri $z$ in nato za nek čas $bT$ v smeri $x$. 
\subparagraph{Hipoteza:}
Predvidevam, da za dovolj majhen $T$ in $b$ (tj. T precej manjši od obhodnega časa), bo rezultat simulacije precesija spina v obliki spirale.

\section{Program za simuliranje}
Za simuliranje uporabljam programski jezik \textit{python} in module \textit{numpy, matplotlib} ter za simuliranje kvantnih sistemov \textit{qutip}. 

\section{Rezultati simulacije}

\section{Zaključek}

\section{Literatura}

\end{document}